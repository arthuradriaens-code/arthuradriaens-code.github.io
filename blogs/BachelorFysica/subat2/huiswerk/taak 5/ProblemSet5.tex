\documentclass[10pt,a4paper,twoside]{article}
\usepackage[english]{babel}
\usepackage{amsmath}
\usepackage{braket}
\usepackage{amssymb,amsfonts,textcomp}
\usepackage{subfig}
\usepackage{graphicx}
\usepackage{float,flafter}
\usepackage{cite}
\usepackage{hyperref}
\usepackage[utf8]{inputenc}
\usepackage{float}
\usepackage{pgfplotstable,filecontents}
\setlength\paperwidth{20.999cm}\setlength\paperheight{29.699cm}\setlength\voffset{-1in}\setlength\hoffset{-1in}\setlength\topmargin{1.499cm}\setlength\headheight{12pt}\setlength\headsep{0cm}\setlength\footskip{1.131cm}\setlength\textheight{25cm}\setlength\oddsidemargin{2.499cm}\setlength\textwidth{15.999cm}
\newcommand{\apj}{The Astrophysical Journal}
\newcommand{\lambdabar}{{\mkern0.75mu\mathchar '26\mkern -9.75mu\lambda}}
\setcounter{section}{5}
\begin{document}
	\begin{center}
		\hrule
		\vspace{.4cm}
		{\bf {\huge Subatomic Physics II: Problem set 5}}
		\vspace{.2cm}
		\\
		{\bf Arthur Adriaens}
		\vspace{.2cm}
		\hrule
	\end{center}
\subsection{Strong coupling constant}
In first order approximation we are given that:
\begin{equation}
	\alpha_s(Q^2) = \frac{\alpha_s(m_Z^2)}{1 + \beta_0\alpha_s(m_Z^2)ln\left(\frac{Q^2}{m_Z^2}\right)}, \text{	where	}\beta_0 = \frac{33-2N_q}{12\pi}
	\label{order1}
\end{equation}
With $N_q$ the number of active quark flavours and $\alpha_s(m_Z^2)\approx0.12$. 
\subsubsection*{Planck scale value of $\alpha_s$}
For this we'll Assume $N_q=6$ (up,down,strange,charm,bottom and top), if we fill in $Q^2 = m_{Pl}^2 = (10^{19}$ GeV$)^2$ and $m_Z = 91.1876\pm0.0021$GeV\cite{Group2020}$\approx 91.19$ GeV (we neglect the uncertainty as it's just an approximation) in equation \ref{order1} we get that $\alpha_s(m_{Pl}^2) \approx 0.02$.\\\\
\noindent
This value is lower compared to $\alpha_s(m_Z^2)$ as $Q^2$ is higher ($10^{38} \gg 91.19^2$), $\alpha_s$ behaves this way because gluons can carry colour themselves, and therefore can also couple to other gluons. The higher $Q^2$ is, the smaller the distances between interacting particles (quarks) and thus the gluons have less space to couple to each other making it so the interacting particles see less charge and thus decreasing $\alpha_s$ (also called antiscreening). There's also an effect which causes screening but this is much weaker than the antiscreening.

\subsubsection*{Find $\Lambda$ in MeV}
A more compact way of writing \ref{order1} is:
\begin{equation}
	\alpha_s(Q^2) = \frac{1}{\beta_0\ln\left(\frac{Q^2}{\Lambda^2}\right)}
\end{equation}
If we assume $N_q = 3$ (i.e only the 3 lowest energy quarks) we find $\Lambda$ by plugging in $\alpha_s(m_z^2)\approx0.12$ (as the original formula is expanded around this point) giving:
\begin{eqnarray}
	\alpha_s(Q^2) &=& \frac{1}{\beta_0\ln\left(\frac{Q^2}{\Lambda^2}\right)}\\
	\ln\left(\frac{Q^2}{\Lambda^2}\right) &=& \frac{1}{\alpha_s(Q^2)\beta_0}\\
	\ln\left(\frac{\Lambda^2}{Q^2}\right) &=& -\frac{1}{\alpha_s(Q^2)\beta_0}\\
	\frac{\Lambda^2}{Q^2} &=& \exp\left(-\frac{1}{\alpha_s(Q^2)\beta_0}\right)\\
	\Lambda^2 &=& Q^2\exp\left(-\frac{1}{\alpha_s(Q^2)\beta_0}\right)\\
	&\stackrel{Q^2 \approx 8315.18\text{ GeV}^2}{\approx}& 73.6\cdot10^3\text{ MeV}^2\\
	 \Lambda &\approx& 271.2 \text{ MeV}
\end{eqnarray}
So we get that $\Lambda \approx 271.2 \text{ MeV}$.
\subsection{Running Quark masses}
Quark masses are also running quantities, in the leading logarithmic QCD approximation we have:
\begin{equation}
	m_q = m_q(m_Z^2)\left[\frac{\alpha_s(Q^2)}{\alpha_s(m_Z^2)}\right]^c, \text{ where } c=\frac{1}{\pi\beta_0}
	\label{masses}
\end{equation}

\subsubsection*{Bottom quark mass at Z-boson mass scale}
The Bare quark masses found in Particles and Nuclei\cite{ParticlesAndNuclei} are:
\begin{center}
	\begin{tabular}{c|c}
		Quark & Mass (MeV)\\
		\hline
		Down & $\approx 300$\\
		Up & $\approx 300$\\
		Strange & $\approx 450$\\
		Charm & $\approx 1,250 - 1,300$\\
		Bottom & $\approx 4,150 - 4,210$\\
		Top & $\approx 172.5\cdot10^3 - 174.5\cdot10^3$\\
	\end{tabular} 
\end{center}
Now due to the fact that the b quark has a mass of about $\approx 4.2\cdot10^3\text{ MeV}$ at this center-of-mass energy it won't be possible to produce any top quarks, and thus we'll take $N_q = 5$.\\\\
We can rewrite equation \ref{masses} as:
\begin{equation}
	m_q(m_Z^2) = m_q\left[\frac{\alpha_s(Q^2)}{\alpha_s(m_Z^2)}\right]^{-c} = m_q\left[\frac{\alpha_s(m_Z^2)}{\alpha_s(Q^2)}\right]^{c}
	\label{massesZ}
\end{equation}
With which we can calculate the mass of the bottom quark at the Z-boson mass scale, we'll start by calculating $\alpha_s(Q^2 = (4.2 \text{ GeV})^2)$ which is $\approx 0.22$, plugging this, as well as $m_b = 4.2$ GeV into equation \ref{massesZ} yields $m_b(m_Z^2) \approx 3.07$\text{ GeV}.

\subsubsection*{Comparing values}
The DELPHI Collaboration measured\cite{Abdallah}:
\begin{equation}
	m_b(m_Z^2) = 2.85 \pm 0.32\text{GeV}
\end{equation}
Which means that our calculation falls inside the confidence interval of the measurement.

\subsubsection*{Experimental determination of c-quark mass running and consistency}
A. Gizhko et al. have experimentally determinated the running of the c-quark mass from HERA deep-inelastic scattering data\cite{RunningCQuarkMass}.\\\\Here it's assumed that with 'are the measurements for both c and b quarks consistent' it's asked to look at the consistency of the measurements with the theory, the measurements of the charm running mass is said to be consistent with the theory in the previously mentioned paper: "The scale dependence of the mass is found to be consistent with QCD expectations". As well as that of the botton (beauty) quark: "The results are found to agree with theoretical predictions treating mass corrections at next-to-leading order"\cite{Abdallah}.
\bibliography{sources}
\bibliographystyle{plain}
\end{document}
