\documentclass[10pt,a4paper,twoside]{article}
\usepackage[english]{babel}
\usepackage{amsmath}
\usepackage{amssymb,amsfonts,textcomp}
\usepackage{subcaption}
\usepackage{graphicx}
\usepackage{float,flafter}
\usepackage{cite}
\usepackage{hyperref}
\usepackage[utf8]{inputenc}
\usepackage{float}
\usepackage{pgfplotstable,filecontents}
\usepackage{tikz-feynman}
\setlength\paperwidth{20.999cm}\setlength\paperheight{29.699cm}\setlength\voffset{-1in}\setlength\hoffset{-1in}\setlength\topmargin{1.499cm}\setlength\headheight{12pt}\setlength\headsep{0cm}\setlength\footskip{1.131cm}\setlength\textheight{25cm}\setlength\oddsidemargin{2.499cm}\setlength\textwidth{15.999cm}
\newcommand{\apj}{The Astrophysical Journal}
\setcounter{section}{8}
\begin{document}
	\begin{center}
		\hrule
		\vspace{.4cm}
		{\bf {\huge Subatomic Physics II: Problem set 8}}
		\vspace{.2cm}
		\\
		{\bf Arthur Adriaens}
		\vspace{.2cm}
		\hrule
	\end{center}
\subsection{Higgs boson properties}
\subsubsection{Dominant Higgs boson production channels at LEP and LHC}
\textbf{Gluon-Gluon Fusion (LHC)}\\
\begin{figure}[h]
	\centering
	\begin{tikzpicture}
	\begin{feynman}
	\vertex (a0) {\(q'\)};
	\vertex[right=3cm of a0] (ai);
	\vertex[below=1cm of ai] (a1);
	\vertex[right=0.5cm of a1] (ci1);
	\vertex[below=0.5cm of ci1] (c1);
	\vertex[right=3cm of ai] (a2) {\(q'\)};
	\vertex[below=4cm of a0] (b0) {\(q\)};
	\vertex[right=3cm of b0] (bi);
	\vertex[above=1cm of bi] (b1);
	\vertex[right=0.5cm of b1] (ci2);
	\vertex[above=0.5cm of ci2] (c2);
	\vertex[above=0.5cm of c2] (cm);
	\vertex[right=1cm of cm] (c3);
	\vertex[right=2cm of c3] (h) {\(h\)};
	\vertex[right=3cm of bi] (b2) {\(q\)};
	\diagram* {
		{[edges=fermion]
			(a0) -- (a1) -- (a2)
		},
		{[edges=fermion]
			(b0) -- (b1) -- (b2)
		},
		(a1) -- [gluon] (c1),
		(b1) -- [gluon] (c2),
		(c1) -- [fermion, edge label= t] (c3) --[fermion, edge label = t] (c2) --[fermion, edge label = t] (c1),
		(c3) -- [scalar] (h)
	};
	\end{feynman}
	\end{tikzpicture}
	\caption{Gluon Gluon fusion}
\end{figure}\\
\\\\
\textbf{Vector Boson Fusion (LHC)}\\
\begin{figure}[h]
	\centering
	\begin{tikzpicture}
	\begin{feynman}
	\vertex (a0) {\(q'\)};
	\vertex[right=3cm of a0] (ai);
	\vertex[below=1cm of ai] (a1);
	\vertex[right=0.5cm of a1] (ci1);
	\vertex[below=1cm of ci1] (c1);
	\vertex[right=3cm of ai] (a2) {\(q\)};
	\vertex[below=4cm of a0] (b0) {\(q\)};
	\vertex[right=3cm of b0] (bi);
	\vertex[above=1cm of bi] (b1);
	\vertex[right=3cm of c1] (h) {\(h\)};
	\vertex[right=3cm of bi] (b2) {\(q'\)};
	\diagram* {
		{[edges=fermion]
			(a0) -- (a1) -- (a2)
		},
		{[edges=fermion]
			(b0) -- (b1) -- (b2)
		},
		(a1) -- [boson, edge label=\(W^+\)] (c1)-- [boson, edge label=\(W^-\)] (b1),
		(c1) -- [scalar] (h)
	};
	\end{feynman}
	\end{tikzpicture}
	\caption{Vector Boson Fusion}
\end{figure}\\
Here one of the W bosons is off-shell.\\
\textbf{Higgs stralung (LEP)}
\begin{figure}[h]
	\centering
	\begin{tikzpicture}
	\begin{feynman}
	\vertex (a0) {\(e^-\)};
	\vertex[below=3cm of a0] (a2) {\(e^+\)};
	\vertex[below=1.5cm of a0] (am);
	\vertex[right=1cm of am] (a1);
	\vertex[right=2cm of a1] (f2);
	\vertex[right=1cm of f2] (fm);
	\vertex[above=1.5cm of fm] (f1);
	\vertex[below=1.5cm of fm] (f3) {\(h\)};
	\diagram* {
		{[edges=fermion]
			(a0) -- (a1) -- (a2)
		},
		(a1) -- [boson, edge label=\(W/Z^*\)] (f2) -- [boson, edge label=\(W/Z\)] (f1), 
		(f2) -- [scalar] (f3)
	};
	\end{feynman}
	\end{tikzpicture}
	\caption{Vector Boson Fusion}
\end{figure}\\
Here the star denotes "off-shell".
\subsubsection{Dominant decay channel at $m_h=125$ GeV}
The dominant decay channel at $m_h=125$ is $H\rightarrow b\bar{b}$:
\begin{figure}[h]
	\centering
	\begin{tikzpicture}
		\begin{feynman}
		\vertex (a0) {\(h\)};
		\vertex[right=2cm of a0] (a1);
		\vertex[right=1cm of a1] (am);
		\vertex[above=1cm of am] (a2) {\(\bar{b}\)};
		\vertex[below=1cm of am] (a3) {\(b\)};
		\diagram*{
		(a0)--[scalar] (a1),
		(a2) -- [fermion] (a1) -- [fermion] (a3),
		};
		\end{feynman}
	\end{tikzpicture}
\end{figure}
\subsubsection{Branching ratio in the "golden channel"}
We can approximate the branching ratio of the feynman diagram under consideration:\\
\begin{figure}[h]
	\centering
	\begin{tikzpicture}
	\begin{feynman}
	\vertex (a0) {\(h\)};
	\vertex[right=2cm of a0] (a1);
	\vertex[right=1cm of a1] (am);
	\vertex[above=1cm of am] (a2);
	\vertex[right=1cm of a2] (a2m);
	\vertex[above=0.5cm of a2m] (f1) {\(l^+\)};
	\vertex[below=0.5cm of a2m] (f2) {\(l^-\)};
	\vertex[below=1cm of am] (a3);
	\vertex[right=1cm of a3] (a3m);
	\vertex[above=0.5cm of a3m] (f3) {\(l^+\)};
	\vertex[below=0.5cm of a3m] (f4) {\(l^-\)};
	\diagram*{
		(a0)--[scalar] (a1),
		(a3) -- [boson, edge label = \(Z^*\)] (a1) -- [boson, edge label = \(Z\)] (a2),
		(f1) -- [fermion] (a2) -- [fermion] (f2),
		(f3) -- [fermion] (a3) -- [fermion] (f4),
	};
	\end{feynman}
	\end{tikzpicture}
\end{figure}\\
As the following product\footnote{individual branching ratio's from PDG\cite{PDG}}:
\begin{align}
	\mathcal{B}r(H\rightarrow l^+l^-l^+l^-) &\approx \mathcal{B}r(H\rightarrow ZZ^*)\times\mathcal{B}r(Z\rightarrow l^+l^-)^2\\
	&\approx 2.7\%\times[\mathcal{B}r(Z\rightarrow e^+e^-) + \mathcal{B}r(Z\rightarrow \mu^+\mu^-)]^2\\
	&\approx 0.014 \%
\end{align}
\subsubsection{Why dit it take longer to find $h\rightarrow \mu^+\mu^-$ than $h \rightarrow \tau^+\tau^-$?}
This is because in a $h\rightarrow l^+l^-$ feynamnn diagram:
\begin{figure}[h]
	\centering
	\begin{tikzpicture}
	\begin{feynman}
	\vertex (a0) {\(h\)};
	\vertex[right=2cm of a0] (a1);
	\vertex[right=1cm of a1] (am);
	\vertex[above=1cm of am] (a2) {\(\bar{\ell}\)};
	\vertex[below=1cm of am] (a3) {\(\ell\)};
	\diagram*{
		(a0)--[scalar] (a1),
		(a2) -- [fermion] (a1) -- [fermion] (a3),
	};
	\end{feynman}
	\end{tikzpicture}
\end{figure}\\
We have that in this vertex $\mathcal{M} \propto m_\ell\frac{g_w}{2 m_w}$, so the decay into 2 $\tau$ leptons is $\frac{\Gamma_\tau}{\Gamma_{\mu}}\propto\frac{|\mathcal{M}_{\tau}|^2}{|\mathcal{M}_{\mu}|^2}\propto\left(\frac{m_\tau}{m_\mu}\right)^2 \approx 282.81$ times more likely then the decay into 2 muons. As such $\approx 283$ times the measurements are needed to get the same $\sigma$ which of course takes longer.
\newpage
\subsection{Fighting misconceptions}
\subsubsection{Higgs bosons are responsible for masses of all other particles}
Particles acquire masses through their interactions with the Higgs field, the boson is just the excitation of this field, a neutral scalar particle which couples to all fermions with a coupling strength proportional to the fermion mass.
\subsubsection{Higgs mechanism gives mass to all massive objects in the Universe}
Most of the mass in composite particles, like protons, nuclei, and atoms, does not come from the Higgs mechanism, but from the binding energy that holds these particles together.
\subsubsection{The Higgs mechanism produces mass, and masses gravitate, therefore the Higgs mechanism is the origin of gravity}
By coupling with the Higgs field a massless particle acquires a certain amount of potential energy and, hence, according to the mass-energy relation, a certain mass. The stronger the coupling, the more massive the particle. In the Higgs mechanism mass is not “generated (/produced)" in the particle by a miraculous creatio ex nihilo, it is only transferred to the particle from the Higgs field\cite{jammer2009concepts}. Together with this I'd like to say, as already discussed, that the Higgs mechanism only accounts for a small percentage of the total mass present in the universe and thus in "gravity".\\
Aside from that, the conclusion on the right side of the sentence is also not completely right as it is not mass that gravitates, it's energy:\\
In Einsteins formulation of general relativity spacetime curves due to energy and momentum being present, we can see this in the field equation:
$R_{\mu \nu}-\frac{1}{2} R g_{\mu \nu}+\Lambda g_{\mu \nu}=\kappa T_{\mu \nu}$
here $T_{\mu \nu}$ is the energy-momentum tensor, which contains the density and flux of all the energy and momentum being present (and thus the masses as $E^2 = m^2 + p^2$ but also light as $E^2 = p^2$). This equation then locks up the metric $g_{\mu\nu}$ which captures all the geometric and causal structure of spacetime, i.e the curvature $\equiv$ gravity.
\subsection{Non-standard Higgses}
\subsubsection{Tree-level production and decay Feynman diagrams of H and H$^\pm$ at the LHC}
\textbf{Production}\\
The extra neutral higgs boson H can be produced by exactly the same processes as the higgs boson h but at higher energies:
\begin{figure}[h]
	\begin{subfigure}{0.49\textwidth}
		\centering
		\begin{tikzpicture}
		\begin{feynman}
		\vertex (a0) {\(q'\)};
		\vertex[right=2cm of a0] (ai);
		\vertex[below=1cm of ai] (a1);
		\vertex[right=0.5cm of a1] (ci1);
		\vertex[below=1cm of ci1] (c1);
		\vertex[right=2cm of ai] (a2) {\(q'\)};
		\vertex[below=4cm of a0] (b0) {\(q\)};
		\vertex[right=2cm of b0] (bi);
		\vertex[above=1cm of bi] (b1);
		\vertex[right=2cm of c1] (h) {\(H\)};
		\vertex[right=2cm of bi] (b2) {\(q\)};
		\diagram* {
			{[edges=fermion]
				(a0) -- (a1) -- (a2)
			},
			{[edges=fermion]
				(b0) -- (b1) -- (b2)
			},
			(a1) -- [boson, edge label=\(Z\)] (c1) --[boson, edge label=\(Z\)] (b1),
			(c1) -- [scalar] (h)
		};
		\end{feynman}
		\end{tikzpicture}
		\caption{Z boson fusion}
	\end{subfigure}
	\begin{subfigure}{0.49\textwidth}
		\centering
		\begin{tikzpicture}
		\begin{feynman}
		\vertex (a0) {\(q'\)};
		\vertex[right=2cm of a0] (ai);
		\vertex[below=1cm of ai] (a1);
		\vertex[right=0.5cm of a1] (ci1);
		\vertex[below=1cm of ci1] (c1);
		\vertex[right=2cm of ai] (a2) {\(q\)};
		\vertex[below=4cm of a0] (b0) {\(q\)};
		\vertex[right=2cm of b0] (bi);
		\vertex[above=1cm of bi] (b1);
		\vertex[right=2cm of c1] (h) {\(H\)};
		\vertex[right=2cm of bi] (b2) {\(q'\)};
		\diagram* {
			{[edges=fermion]
				(a0) -- (a1) -- (a2)
			},
			{[edges=fermion]
				(b0) -- (b1) -- (b2)
			},
			(a1) -- [boson, edge label=\(W^+\)] (c1)-- [boson, edge label=\(W^-\)] (b1),
			(c1) -- [scalar] (h)
		};
		\end{feynman}
		\end{tikzpicture}
		\caption{Vector Boson Fusion}
	\end{subfigure}
\end{figure}\\
\newpage
\noindent
Now for the production of H$^{\pm}$ we can take a look at the processes necessary to produce $W^\pm$ and just increase the COM energy:
\begin{figure}[h]
	\begin{subfigure}{0.49\textwidth}
		\centering
		\begin{tikzpicture}
		\begin{feynman}
		\vertex (a0);
		\vertex[below=3cm of a0] (a2);
		\vertex[below=1.5cm of a0] (am);
		\vertex[right=1cm of am] (a1);
		\vertex[right=2cm of a1] (f2) {\(H^\pm\)};
		\diagram* {
			(a0) -- [fermion, edge label=\(q\)] (a1) -- [fermion, edge label= \(\bar{q}'\)] (a2),
			(a1) -- [scalar] (f2),
		};
		\end{feynman}
		\end{tikzpicture}
		\caption{Quark-Antiquark fusion}
	\end{subfigure}
	\begin{subfigure}{0.49\textwidth}
		\centering
		\begin{tikzpicture}
		\begin{feynman}
		\vertex (a0) {\(q\)};
		\vertex[right=2cm of a0] (ai);
		\vertex[below=1cm of ai] (a1);
		\vertex[right=0.5cm of a1] (ci1);
		\vertex[below=1cm of ci1] (c1);
		\vertex[right=2cm of ai] (a2) {\(q'\)};
		\vertex[below=4cm of a0] (b0) {\(q\)};
		\vertex[right=2cm of b0] (bi);
		\vertex[above=1cm of bi] (b1);
		\vertex[right=2cm of c1] (h) {\(H^\pm\)};
		\vertex[right=2cm of bi] (b2) {\(q\)};
		\diagram* {
			{[edges=fermion]
				(a0) -- (a1) -- (a2)
			},
			{[edges=fermion]
				(b0) -- (b1) -- (b2)
			},
			(a1) -- [boson, edge label=\(W^\pm\)] (c1)-- [boson, edge label=\(Z\)] (b1),
			(c1) -- [scalar] (h)
		};
		\end{feynman}
		\end{tikzpicture}
		\caption{ZW Boson Fusion}
	\end{subfigure}
\end{figure}\\
\textbf{Decay}\\
The extra neutral H boson will decay in the same way as the higgs boson (with more energy and no fermions):
\begin{figure}[h]
	\begin{subfigure}{0.49\textwidth}
		\centering
		\begin{tikzpicture}
		\begin{feynman}
		\vertex (a0) {\(H\)};
		\vertex[right=2cm of a0] (a1);
		\vertex[right=1cm of a1] (am);
		\vertex[above=1cm of am] (a2) {\(W^+\)};
		\vertex[below=1cm of am] (a3) {\(W^-\)};
		\diagram*{
			(a0)--[scalar] (a1),
			(a2) -- [boson] (a1) -- [boson] (a3),
		};
		\end{feynman}
		\end{tikzpicture}
	\end{subfigure}
	\begin{subfigure}{0.49\textwidth}
		\centering
		\begin{tikzpicture}
		\begin{feynman}
		\vertex (a0) {\(H\)};
		\vertex[right=2cm of a0] (a1);
		\vertex[right=1cm of a1] (am);
		\vertex[above=1cm of am] (a2) {\(Z\)};
		\vertex[below=1cm of am] (a3) {\(Z\)};
		\diagram*{
			(a0)--[scalar] (a1),
			(a2) -- [boson] (a1) -- [boson] (a3),
		};
		\end{feynman}
		\end{tikzpicture}
	\end{subfigure}
\end{figure}\\
And the charged ones will decay as:
\begin{figure}[h]
	\begin{subfigure}{0.49\textwidth}
		\centering
		\begin{tikzpicture}
		\begin{feynman}
		\vertex (a0) {\(H^\pm\)};
		\vertex[right=2cm of a0] (a1);
		\vertex[right=1cm of a1] (am);
		\vertex[above=1cm of am] (a2) {\(q\)};
		\vertex[below=1cm of am] (a3) {\(\bar{q}'\)};
		\diagram*{
			(a0)--[boson] (a1),
			(a2) -- [fermion] (a1) -- [fermion] (a3),
		};
		\end{feynman}
		\end{tikzpicture}
	\end{subfigure}
	\begin{subfigure}{0.49\textwidth}
		\centering
		\begin{tikzpicture}
		\begin{feynman}
		\vertex (a0) {\(H^\pm\)};
		\vertex[right=2cm of a0] (a1);
		\vertex[right=1cm of a1] (am);
		\vertex[above=1cm of am] (a2) {\(Z\)};
		\vertex[below=1cm of am] (a3) {\(W^\pm\)};
		\diagram*{
			(a0)--[boson] (a1),
			(a2) -- [boson] (a1) -- [boson] (a3),
		};
		\end{feynman}
		\end{tikzpicture}
	\end{subfigure}
\end{figure}\\
\subsubsection{two-photon decay of the Standard-Model-like boson $h\rightarrow\gamma\gamma$}
There'll be another contribution to the matrix element $\mathcal{M}$ of the decay mode, aside from the usual ones:
\begin{figure}[H]
	\begin{subfigure}{0.49\textwidth}
		\centering
		\begin{tikzpicture}
		\begin{feynman}
		\vertex (a0) {\(h\)};
		\vertex[right=2cm of a0] (a1);
		\vertex[right=1cm of a1] (am);
		\vertex[above=1cm of am] (a2);
		\vertex[below=1cm of am] (a3);
		\vertex[right=1cm of a2] (f1) {\(\gamma\)};
		\vertex[right=1cm of a3] (f2) {\(\gamma\)};	
		\diagram*{
			(a0)--[scalar] (a1),
			(a3) -- [fermion, edge label=\(f\)] (a1) -- [fermion, edge label=\(f\)] (a2) -- [fermion, edge label=\(f\)] (a3),
			(a2) -- [boson] (f1),
			(a3) -- [boson] (f2)
		};
		\end{feynman}
		\end{tikzpicture}
	\end{subfigure}
	\begin{subfigure}{0.49\textwidth}
		\centering
		\begin{tikzpicture}
		\begin{feynman}
		\vertex (a0) {\(h\)};
		\vertex[right=2cm of a0] (a1);
		\vertex[right=1cm of a1] (am);
		\vertex[above=1cm of am] (a2);
		\vertex[below=1cm of am] (a3);
		\vertex[right=1cm of a2] (f1) {\(\gamma\)};
		\vertex[right=1cm of a3] (f2) {\(\gamma\)};	
		\diagram*{
			(a0)--[scalar] (a1),
			(a3) -- [boson, edge label=\(W\)] (a1) -- [boson, edge label=\(W\)] (a2) -- [boson, edge label=\(W\)] (a3),
			(a2) -- [boson] (f1),
			(a3) -- [boson] (f2)
		};
		\end{feynman}
		\end{tikzpicture}
	\end{subfigure}
\end{figure}
\noindent
We'll now also have this possible Feynman diagram:
\begin{figure}[H]
	\centering
	\begin{tikzpicture}
	\begin{feynman}
	\vertex (a0) {\(h\)};
	\vertex[right=2cm of a0] (a1);
	\vertex[right=1cm of a1] (am);
	\vertex[above=1cm of am] (a2);
	\vertex[below=1cm of am] (a3);
	\vertex[right=1cm of a2] (f1) {\(\gamma\)};
	\vertex[right=1cm of a3] (f2) {\(\gamma\)};	
	\diagram*{
		(a0)--[scalar] (a1),
		(a3) -- [boson, edge label=\(H\)] (a1) -- [boson, edge label=\(H\)] (a2) -- [boson, edge label=\(H\)] (a3),
		(a2) -- [boson] (f1),
		(a3) -- [boson] (f2)
	};
	\end{feynman}
	\end{tikzpicture}
\end{figure}
\bibliography{sources}
\bibliographystyle{plain}
\end{document}
