\documentclass[10pt,a4paper,twoside]{article}
\usepackage[english]{babel}
\usepackage{amsmath}
\usepackage{braket}
\usepackage{amssymb,amsfonts,textcomp}
\usepackage{subfig}
\usepackage{graphicx}
\usepackage{float,flafter}
\usepackage{cite}
\usepackage{hyperref}
\usepackage[utf8]{inputenc}
\usepackage{float}
\usepackage{pgfplotstable,filecontents}
\setlength\paperwidth{20.999cm}\setlength\paperheight{29.699cm}\setlength\voffset{-1in}\setlength\hoffset{-1in}\setlength\topmargin{1.499cm}\setlength\headheight{12pt}\setlength\headsep{0cm}\setlength\footskip{1.131cm}\setlength\textheight{25cm}\setlength\oddsidemargin{2.499cm}\setlength\textwidth{15.999cm}
\newcommand{\apj}{The Astrophysical Journal}
\newcommand{\lambdabar}{{\mkern0.75mu\mathchar '26\mkern -9.75mu\lambda}}
\setcounter{section}{4}
\begin{document}
	\begin{center}
		\hrule
		\vspace{.4cm}
		{\bf {\huge Subatomic Physics II: Problem set 4}}
		\vspace{.2cm}
		\\
		{\bf Arthur Adriaens}
		\vspace{.2cm}
		\hrule
	\end{center}
\section*{HERA particle collider at DESY in Hamburg}
\subsection{$Q^2$, x and y in terms of $E_e, E_p, E_0$ and $\theta_e$ }
We have the following four-vectors:
\begin{eqnarray}
	P_e &\approx& (30GeV,30GeV,0,0)\\
	P_p &\approx& (900GeV,-900GeV,0,0)\\
	P_e' &\approx& (E_0, E_0\sin(\theta_c), E_0\cos(\theta_c))
\end{eqnarray}
Now per definition $Q^2$ is given by\cite{ParticlesAndNuclei}:
\begin{equation}
	Q^2 := -(P_e - P_e')^2 \approx 2(EE' - |\boldsymbol{p}||\boldsymbol{p'}|\cos(\theta_c)) = 2(E_eE_0 - E_eE_0\cos(\theta_c)) = 4E_eE_0\sin^2\left(\frac{\theta_c}{2}\right)
\end{equation}
And the Bjorken scaling variable x is given by:
\begin{equation}
	x := \frac{Q^2}{2Pq} = \frac{4E_eE_0\sin^2\left(\frac{\theta_c}{2}\right)}{2P_p(P_e-P_e')} \approx \frac{4E_eE_0\sin^2\left(\frac{\theta_c}{2}\right)}{4E_eE_p - 2E_0E_p(1 + \sin(\theta_c))}
\end{equation}
Where the approximation $|\boldsymbol{p_p}| \approx E_p$ was used. Now finally the inelasticity y is given by:
\begin{equation}
	y := \frac{Pq}{Pp} \approx \frac{2E_eE_p - E_0E_p(1 + \sin(\theta_c))}{2E_eE_p} = 1 - \frac{E_0E_p(1 + \sin(\theta_c))}{2E_eE_p} 
\end{equation}
\subsection{Variables $Q^2$, x, y and $W^2$ in terms of the Mandelstam variables s,t and u}
The Mandelstam variables are\cite{Thomson}:
\begin{eqnarray}
	s &=& (p_1 + p_2)^2 = (p_3 + p_4)^2\\
	t &=& (p_1 - p_3)^2 = (p_2-p_4)^2\\
	u &=& (p_1 - p_4)^2 = (p_2 - p_3)^2
\end{eqnarray}
naming the initial electron and proton number 1 and 2 respectively and the final electron and X numbers 3 and 4 we can proceed.\\
$Q^2$ is by far the easiest one as $Q^2 = -(P_e-P_e')^2 = -(P_1-P_3)^2 = -t$.\\
Now to write x in terms of the Mandelstam variables we'll take a look at Pq:
\begin{equation}
	Pq = P_2(P_1-P_3) = P_2P_1 - P_2P_3
\end{equation}
Now we have that
\begin{equation}
	s + u = P_1^2 + 2P_2^2 + P_3^2 + 2(P_1P_2 - P_2P_3) \Rightarrow \frac{s + u}{2} = m_e^2 + m_p^2 + P_2P_1 - P_2P_3 \approx m_p^2 + P_2P_1 - P_2P_3
\end{equation}
And thus:
\begin{equation}
	x = -\frac{t}{s+u-2m_p^2}
\end{equation}
Now to get the inelasticity y we'll first look at Pp, i.e:
\begin{equation}
	s = P_1^2 + P_2^2 + 2P_1P_2 \approx m_p^2 + 2P_1P_2 \rightarrow Pp = P_2P_1 \approx \frac{s-m_p^2}{2}
\end{equation}
From which we have:
\begin{equation}
	y = \frac{s+u-2m_p^2}{s-m_p^2}
\end{equation}
Now lastly $W^2$ is the squared invariant mass and it is calculated from the four momenta of the exchanged photon q and of the incoming proton P:
\begin{eqnarray}
	W^2 = (P + q)^2 = (P_p + (P_e - P_e'))^2 &=& P_1^2 + P_2^2 + P_3^2 + 2(P_1P_2 - P_2P_3 - P_1P_3)\\
	&\approx& m_p^2 + 2(P_1P_2 - P_2P_3 - P_1P_3)\\
	&=& s+u+t - 2m_p^2
\end{eqnarray}
as $s+u+t = 3m_p^2 + 2(P_1P_2 - P_2P_3 - P_1P_3)$.\\
So summing up:
\begin{itemize}
	\item $Q^2 = -t$
	\item $x = -\frac{t}{s+u-2m_p^2}$
	\item $y = \frac{s+u-2m_p^2}{s-m_p^2}$
	\item $W^2 = s+u+t - 2m_p^2$
\end{itemize}
\subsection{Energy reached by an electron beam hitting a solid proton target to have the same centre-of-mass energy as HERA}
For this we'll first calculate the center of mass energy in this situation:
\begin{equation}
	s = (P_p + P_e)^2 \approx m_p^2 + 2*(30*900 \text{GeV}^2 + 900*30 \text{GeV}^2) \Rightarrow \sqrt{s} \approx 328.64\text{GeV}
\end{equation}
Where I have used 938 MeV as the proton mass\cite{mohr_2016}.\\
Now if the proton would be standing still then:
\begin{equation}
	s = m_p^2 + 2m_pE_e = 108000.88\text{GeV}^2\Rightarrow E_e = \frac{108000.88\text{GeV}^2- m_p^2}{2m_p} = 57569.3\text{GeV}
\end{equation}
So the electron would need an energy of 58 TeV to match the center of mass energy.

\section*{Structure functions}
\subsection{Sum rules}
The structure functions of the proton and neutron for electron-nucleon inelastic scattering are respectively given by:
\begin{eqnarray}
	F_2^{e,p}(x) &=& x\cdot\left[\frac{4}{9}(u_v^p + u_s^p + \bar{u}_s^p) + \frac{1}{9}(d_v^p + d_s^p + \bar{d}_s^p) + \frac{1}{9}(s^p_s + \bar{s}^p_s)\right]\\
	F_2^{e,n}(x) &=& x\cdot\left[\frac{4}{9}(u_v^n + u_s^n + \bar{u}_s^n) + \frac{1}{9}(d_v^n + d_s^n + \bar{d}_s^n) + \frac{1}{9}(s^n_s + \bar{s}^n_s)\right]
\end{eqnarray}
Where $u^{p,n}_v(x)$ denotes the distribution of valence u-quarks in the proton and neutron respectively and the s in $u_s^{p,n}(x)$ denoting sea-quark. Now because of isospin symmetry
\begin{eqnarray}
	u^p_{v,s}(x) &=& d^n_{v,s}(x) := u_{v,s}(x)\\
	d^p_{v,s}(x) &=& u^n_{v,s}(x) := d_{v,s}(x)
\end{eqnarray}
As such we get that the difference of the proton and neutron structure functions is given by:
\begin{equation}
	F^{e,p}_2 - F^{e,p}_2 = x\cdot\left[\frac{1}{3}(u_v(x) - d_v(x)) + \frac{2}{3}(\bar{u}_s(x) - \bar{d}_s(x))\right]
\end{equation}
The parton distribution function is defined in such a way that $u^p(x)d x$ represents the number of up-quarks within the proton with momentum fraction between x and x+dx, because of we have that
\begin{equation}
	\int_{0}^{1}u_v^p = 2
\end{equation}
Now evaluating Gottfried's integral:
\begin{equation}
	S_G = \int_{0}^{1}\frac{dx}{x}(F^{e,p}_2 - F^{e,n}_2) = \int_{0}^1\left(\frac{1}{3}(u_v - d_v) + \frac{2}{3}(\bar{u}_s(x) - \bar{d}_s(x))\right) = \frac{1}{3}(2-1) + \frac{2}{3}\int_{0}^1(\bar{u}_s(x) - \bar{d}_s(x))
\end{equation}
Now if we also take the amount of sea quarks to be equal because of isospin invariance we finally expect:
\begin{equation}
	S_G = \int_{0}^{1}\frac{dx}{x}(F^{e,p}_2 - F^{e,n}_2) = \frac{1}{3}
\end{equation}
The structure functions of the proton and neutron for neutrino-nucleon interactions are given by:
\begin{eqnarray}
	F_2^{\nu,p}(x) &=& 2x\cdot\left[d(x) + \bar{u}(x)\right]\\
	F_2^{\nu,n}(x) &=& 2x\cdot\left[u(x) + \bar{d}(x)\right]
\end{eqnarray}
Adler's sum rule is thus:
\begin{eqnarray}
	S_A = \int_{0}^{1}\frac{dx}{x}(F^{\nu,n}_2 - F^{\nu,p}_2) &=& 2\int_0^1(u(x) + \bar{d}(x) - d(x) - \bar{u}(x))\\
	&=& 2\int_0^1(u_v(x) + u_s(x) + \bar{d}_s(x) - d_v(x) - d_s(x) - \bar{u}_s(x))\\
	&=& 2\int_0^1(u_v(x) - d_v(x)) = 2
\end{eqnarray}
\subsection{Why is the Adler sum rule better satisfied than the Gottfried sum rule?}
A possible flaw with the Gottfried Sum Rule is that it's assumed (contrary to Adler's rule in which the sea quarks vanish naturally) that $\bar{d}_s(x) = \bar{u}_s(x)$ and thus that the quark-antiquark sea is symmetric which doesn't have to be the case.
\bibliography{sources}
\bibliographystyle{plain}
\end{document}
