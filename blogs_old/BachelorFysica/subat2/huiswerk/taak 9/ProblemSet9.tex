\documentclass[10pt,a4paper,twoside]{article}
\usepackage[english]{babel}
\usepackage{amsmath}
\usepackage{amssymb,amsfonts,textcomp}
\usepackage{subcaption}
\usepackage{graphicx}
\usepackage{float,flafter}
\usepackage{cite}
\usepackage{hyperref}
\usepackage[utf8]{inputenc}
\usepackage{float}
\usepackage{pgfplotstable,filecontents}
\usepackage{tikz-feynman}
\setlength\paperwidth{20.999cm}\setlength\paperheight{29.699cm}\setlength\voffset{-1in}\setlength\hoffset{-1in}\setlength\topmargin{1.499cm}\setlength\headheight{12pt}\setlength\headsep{0cm}\setlength\footskip{1.131cm}\setlength\textheight{25cm}\setlength\oddsidemargin{2.499cm}\setlength\textwidth{15.999cm}
\newcommand{\apj}{The Astrophysical Journal}
\setcounter{section}{8}
\begin{document}
	\begin{center}
		\hrule
		\vspace{.4cm}
		{\bf {\huge Subatomic Physics II: Problem set 9}}
		\vspace{.2cm}
		\\
		{\bf Arthur Adriaens}
		\vspace{.2cm}
		\hrule
	\end{center}
\section{CKM parameters}
\subsection{Interactions in $\Lambda \rightarrow pe^-\bar{\nu}_e$}
In this interaction we have the decay of a strange quark into an up quark by emittance of a $W^-$ boson (as quarks can only change via weak decay):
\begin{figure}[h]
	\centering
	\begin{tikzpicture}
	\begin{feynman}
	\vertex (a1) {\(u\)};
	\vertex[above=1cm of a1] (a0) {\(s\)};
	\vertex[below=1cm of a1] (a2) {\(d\)};
	\vertex[right=3cm of a0] (m0);
	\vertex[right=2.5cm of a0] (i);
	\vertex[above=0.1cm of i] (tekst)  {\(V_{us}\)};
	\vertex[right=1.5cm of m0] (m1);
	\vertex[above=2cm of m1] (f3);
	\vertex[right=2cm of f3] (lm);
	\vertex[above=0.5cm of lm] (f4) {\(e^-\)};
	\vertex[below=0.5cm of lm] (f5) {\(\bar{\nu}_e\)};
	\vertex[right=3cm of m0] (f0) {\(u\)};
	\vertex[right=6cm of a1] (f1) {\(u\)};
	\vertex[right=6cm of a2] (f2) {\(d\)};
	\diagram* {
			(a0) --[fermion] (m0) --[fermion] (f0),
		{[edges=fermion]
			(a1) -- (f1)
		},
		{[edges=fermion]
			(a2) -- (f2)
		},
		{[edges=boson, edge label=\(W^-\)]
			(m0) -- (f3)
		},
		{[edges=fermion]
			(f5) -- (f3) -- (f4)
		},
	};
	\end{feynman}
	\end{tikzpicture}
	\caption{$\Lambda \rightarrow pe^-\bar{\nu}_{e}$}
\end{figure}\\
Now as
\begin{align}
	\Gamma\left(\Lambda \rightarrow pe^-\bar{\nu}_{e}\right)&=\frac{G_{F}^{2} m_{\Lambda}^{5}}{192 \pi^{3}}\left(1+\delta_{e}^{\Lambda}\right)\left|V_{u s}\right|^{2}\\
	&= \frac{B\left(\Lambda \rightarrow pe^-\bar{\nu}_{e}\right)}{\tau_{\Lambda}}
\end{align}
we have that
\begin{equation}
	|V_{us}| \propto B^{\frac{1}{2}}\tau_{\Lambda}^{-\frac{1}{2}} \implies \frac{\delta |V_{us}|}{|V_{us}|} = \frac{1}{2}\left(\frac{\delta B}{B} + \frac{\delta \tau_{\Lambda}}{\tau_{\Lambda}}\right) = 0.0122
\end{equation}
I.e a 1.22\% relative error.
\subsection{$K^0 \leftrightarrow \bar{K}^0$}
\begin{figure}[h]
	\centering
	\begin{subfigure}{0.49\textwidth}
		\centering
		\begin{tikzpicture}
		\begin{feynman}
		\vertex (a1) {\(d\)};
		\vertex[below=2cm of a1] (a2) {\(\bar{s}\)};
		\vertex[right=2cm of a1] (Vcd);
		\vertex[above=0.1cm of Vcd] (tekstVcd) {\(V_{cd}\)};
		\vertex[right=2cm of a2] (Vcs);
		\vertex[below=0.1cm of Vcs] (tekstVcs) {\(V_{cs}^*\)};
		\vertex[right=2cm of Vcd] (VcsR);
		\vertex[above=0.1cm of VcsR] (tekstVcsR) {\(V_{cs}^*\)};
		\vertex[right=2cm of Vcs] (VcdR);
		\vertex[below=0.1cm of VcdR] (tekstVcdR) {\(V_{cd}\)};
		\vertex[right=2cm of VcsR] (f1) {\(s\)};
		\vertex[right=2cm of VcdR] (f2) {\(\bar{d}\)};
		\diagram* {
			(a1) --[fermion] (Vcd) --[fermion, edge label=\(c\)] (Vcs) --[fermion] (a2),
			(f2) --[fermion] (VcdR) --[fermion, edge label=\(c\)] (VcsR) --[fermion] (f1),
			(Vcd) -- [boson] (VcsR),
			(Vcs) -- [boson] (VcdR),
		};
		\end{feynman}
		\end{tikzpicture}
		\caption{$K^0 \rightarrow \bar{K}^0$}
	\end{subfigure}
	\begin{subfigure}{0.49\textwidth}
		\centering
		\begin{tikzpicture}
		\begin{feynman}
		\vertex (a1) {\(s\)};
		\vertex[below=2cm of a1] (a2) {\(\bar{d}\)};
		\vertex[right=2cm of a1] (Vcd);
		\vertex[above=0.1cm of Vcd] (tekstVcd) {\(V_{cs}\)};
		\vertex[right=2cm of a2] (Vcs);
		\vertex[below=0.1cm of Vcs] (tekstVcs) {\(V_{cd}^*\)};
		\vertex[right=2cm of Vcd] (VcsR);
		\vertex[above=0.1cm of VcsR] (tekstVcsR) {\(V_{cd}^*\)};
		\vertex[right=2cm of Vcs] (VcdR);
		\vertex[below=0.1cm of VcdR] (tekstVcdR) {\(V_{cs}\)};
		\vertex[right=2cm of VcsR] (f1) {\(d\)};
		\vertex[right=2cm of VcdR] (f2) {\(\bar{s}\)};
		\diagram* {
			(a1) --[fermion] (Vcd) --[fermion, edge label=\(c\)] (Vcs) --[fermion] (a2),
			(f2) --[fermion] (VcdR) --[fermion, edge label=\(c\)] (VcsR) --[fermion] (f1),
			(Vcd) -- [boson] (VcsR),
			(Vcs) -- [boson] (VcdR),
		};
		\end{feynman}
		\end{tikzpicture}
		\caption{$\bar{K}^0 \rightarrow K^0$}
	\end{subfigure}
\end{figure}\noindent
We can see in the left figure (right figure is completely analogous) that we have a term proportional to $V_{cd}$ and $V_{cs}^*$ in the 2 left-hand side vertices of the box-diagram which is also proportional to $V_{cd}^\dagger$ and $(V_{cs}^*)^\dagger$ as seen in the right-hand side of the diagram, yielding an interaction proportional to:
\begin{equation}
	\propto V_{cd}V_{cs}^*V_{cd}^\dagger(V_{cs}^*)^\dagger = V_{cd}V_{cd}^\dagger V_{cs}^*(V_{cs}^*)^\dagger = |V_{cs}|^2|V_{cd}|^2 \approx 0.218
\end{equation}
The contribution of the virtual u quark can be neglected in the contribution of the oscillation as $\Delta m \propto m_q^2$ and $m_c^2 \gg m_u^2$. The contribution of the virtual t quark can be neglected as $|V_{td}|^2|V_{ts}|^2 \ll |V_{cs}|^2|V_{cd}|^2$, concretely, defining $\Delta m_q$ as denoting the contribution of the virtual quark q:
\begin{equation}
	\Delta m_q = m_q^2|V_{qd}|^2|^2V_{qs}|^2
\end{equation}
We have that
\begin{align}
	\Delta m_c &\approx  0.3\\
	&\gg  \Delta m_t \approx  0.06\\
	&\gg \Delta m_u \approx 0.002\\
\end{align}
We'll now try to find the Cabbibo angle in the Cabibbo approximation of the CKM matrix:
\begin{equation}
	\left( {\begin{array}{c}
		d'\\
		s'\\
		\end{array} } \right) = \left( {\begin{array}{cc}
		\cos\theta_c & \sin\theta_c\\
		-\sin\theta_c & \cos\theta_c \\
		\end{array} } \right) \left( {\begin{array}{c}
		d\\
		s\\
		\end{array} } \right)
\end{equation}
as we have that
\begin{equation}
	\Delta m = 3.484\times10^{-6} eV = \frac{G_F^2}{4\pi^2}f_k^2m_Km_c^2|V_{cd}|^2|V_{cs}|^2 \approx \frac{G_F^2}{4\pi^2}f_k^2m_Km_c^2\cos^2\theta_c\sin^2\theta_c
\end{equation}
Filling in $f_K\approx0.16$GeV, $m_c \approx 1.4$GeV, $G_F = 1.166\times10^{-5}$ GeV$^{-2}$ \cite{Group2020} and $m_K \approx 0.497611$ GeV we get:
\begin{equation}
	3.484\times10^{-6} \approx 8.6\times10^{-5} \cos^2\theta_c\sin^2\theta_c = 8.6\times10^{-5} (1-\sin^2\theta_c)\sin^2\theta_c 
\end{equation}
And solving for $\sin\theta_c$ yields $\theta_c \approx 0.207$ ($\approx 12$°).
\subsection{Determine the ratio $\frac{|V_{cd}|}{V_{cs}}$}
The relevant feynmann diagrams are:
\begin{figure}[h]
	\centering
	\begin{tikzpicture}
	\begin{feynman}
	\vertex (a1) {\(\bar{u}\)};
	\vertex[above=1cm of a1] (a0) {\(c\)};
	\vertex[right=3cm of a0] (m0);
	\vertex[right=2.5cm of a0] (i);
	\vertex[above=0.1cm of i] (tekst)  {\(V_{cs}\)};
	\vertex[right=1cm of m0] (m1);
	\vertex[above=2cm of m1] (f3);
	\vertex[right=2cm of f3] (lm);
	\vertex[above=0.5cm of lm] (f4) {\(\nu_{\mu}\)};
	\vertex[below=0.5cm of lm] (f5) {\(\mu^+\)};
	\vertex[right=3cm of m0] (f0) {\(s\)};
	\vertex[right=6cm of a1] (f1) {\(\bar{u}\)};
	\diagram* {
		(a0) --[fermion] (m0) --[fermion] (f0),
		{[edges=fermion]
			(a1) -- (f1)
		},
		{[edges=boson, edge label=\(W^+\)]
			(m0) -- (f3)
		},
		{[edges=fermion]
			(f5) -- (f3) -- (f4)
		},
	};
	\end{feynman}
	\end{tikzpicture}
	\caption{$D^0 \rightarrow K^-\mu^+\nu_{\mu}$}
\end{figure}\\
And
\begin{figure}[h]
	\centering
	\begin{tikzpicture}
	\begin{feynman}
	\vertex (a1) {\(\bar{u}\)};
	\vertex[above=1cm of a1] (a0) {\(c\)};
	\vertex[right=3cm of a0] (m0);
	\vertex[right=2.5cm of a0] (i);
	\vertex[above=0.1cm of i] (tekst)  {\(V_{cd}\)};
	\vertex[right=1cm of m0] (m1);
	\vertex[above=2cm of m1] (f3);
	\vertex[right=2cm of f3] (lm);
	\vertex[above=0.5cm of lm] (f4) {\(\nu_{\mu}\)};
	\vertex[below=0.5cm of lm] (f5) {\(\mu^+\)};
	\vertex[right=3cm of m0] (f0) {\(d\)};
	\vertex[right=6cm of a1] (f1) {\(\bar{u}\)};
	\diagram* {
		(a0) --[fermion] (m0) --[fermion] (f0),
		{[edges=fermion]
			(a1) -- (f1)
		},
		{[edges=boson, edge label=\(W^+\)]
			(m0) -- (f3)
		},
		{[edges=fermion]
			(f5) -- (f3) -- (f4)
		},
	};
	\end{feynman}
	\end{tikzpicture}
	\caption{$D^0 \rightarrow \pi^-\mu^+\nu_{\mu}$}
\end{figure}\\
If we consider the pi-meson and kaon to have the same mass, they'll have the same $\tau$ meaning that $B \propto |V|^2$ and thus:
\begin{equation}
	\frac{|V_{cd}|}{|V_{cs}|} = \sqrt{\frac{B(D^0 \rightarrow \pi^-\mu^+\nu_{\mu})}{B(D^0 \rightarrow K^-\mu^+\nu_{\mu})}} = 0.2798
\end{equation}
\subsection{$B^0$ meson lifetime change if $m_t \approx m_c$}
Considering the decay of $B_0$ we have $\tau \propto |V_{cb}|^{-2}$ in the primary decay mode but if the charm quark would have around the same mass as the top quark then we could expect a decay where the $\bar{b}$ quark becomes a $\bar{t}$, this would result in a $|V_{tb}|^2$ contribution and thus the lifetime would change by a factor $\frac{|V_{cb}|^2}{|V_{tb}|^2} = 0.0016$, i.e $B^0$ would live approximately 625 times shorter (it's assumed that the change in mass doesn't affect $|V_{tb}|^2$).
\subsection{measuring $V_{ub}$}
For this we can take a look at the ratio branching ratio's of $B^0$, as:
\begin{equation}
	\frac{B\left(B^{0} \rightarrow \pi^{-} e^{+} \nu_{e}\right)}{\tau_B}=\frac{G_{F}^{2} m_{B}^{5}}{192 \pi^{3}}\left(1+\delta_{e}^{B}\right)\left|V_{u b}\right|^{2}
\end{equation}
And
\begin{equation}
\frac{B\left(B^{0} \rightarrow D^{-} e^{+} \nu_{e}\right)}{\tau_B}=\frac{G_{F}^{2} m_{B}^{5}}{192 \pi^{3}}\left(1+\delta_{e}^{B}\right)\left|V_{cb}\right|^{2}
\end{equation}
We have that
\begin{equation}
	\frac{B\left(B^{0} \rightarrow \pi^{-} e^{+} \nu_{e}\right)}{B\left(B^{0} \rightarrow D^{-} e^{+} \nu_{e}\right)} = \frac{\left|V_{u b}\right|^{2}}{\left|V_{cb}\right|^{2}}
\end{equation}
And thus if we measure the ratio of these branching ratio's and fill in $|V_{cb}|^2$ we'll get $|V_{ub}|$
\bibliography{sources}
\bibliographystyle{plain}
\end{document}
