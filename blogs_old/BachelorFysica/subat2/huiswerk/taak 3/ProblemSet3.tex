\documentclass[10pt,a4paper,twoside]{article}
\usepackage[english]{babel}
\usepackage{amsmath}
\usepackage{braket}
\usepackage{amssymb,amsfonts,textcomp}
\usepackage{subfig}
\usepackage{graphicx}
\usepackage{float,flafter}
\usepackage{cite}
\usepackage{hyperref}
\usepackage[utf8]{inputenc}
\usepackage{float}
\usepackage{pgfplotstable,filecontents}
\setlength\paperwidth{20.999cm}\setlength\paperheight{29.699cm}\setlength\voffset{-1in}\setlength\hoffset{-1in}\setlength\topmargin{1.499cm}\setlength\headheight{12pt}\setlength\headsep{0cm}\setlength\footskip{1.131cm}\setlength\textheight{25cm}\setlength\oddsidemargin{2.499cm}\setlength\textwidth{15.999cm}
\newcommand{\apj}{The Astrophysical Journal}
\newcommand{\lambdabar}{{\mkern0.75mu\mathchar '26\mkern -9.75mu\lambda}}
\begin{document}
	\begin{center}
		\hrule
		\vspace{.4cm}
		{\bf {\huge Subatomic Physics II: Problem set 3}}
		\vspace{.2cm}
		\\
		{\bf Arthur Adriaens}
		\vspace{.2cm}
		\hrule
	\end{center}
\section{SU(2)}
\subsection{$\frac{\sigma(\pi^0d \rightarrow pn)}{\sigma(\pi^+d \rightarrow pp)}$}
For this we can take a look at the angular momenta\cite{ParticlesAndNuclei}:
\begin{center}
	\begin{tabular}{ c||ccc } 
		& J & $I$  & $I_3$\\ 
		\hline
		$\pi^0$ & 0 & 1 & 0 \\ 
		$\pi^+$ & 0 & 1 & 1 \\
		p & 1/2 & 1/2 & 1/2\\
		n & 1/2 & 1/2 & -1/2
	\end{tabular}
\end{center}
Now taking a look at the isospins of the interactions we have the following:\\
If we couple p and n we have $\ket{\frac{1}{2},\frac{1}{2}}\otimes\ket{\frac{1}{2},-\frac{1}{2}} = \sqrt{\frac{1}{2}}\ket{1,0} + \sqrt{\frac{1}{2}}\ket{0,0}$ and if we couple p and p we have $\ket{\frac{1}{2},\frac{1}{2}}\otimes\ket{\frac{1}{2},\frac{1}{2}} = \ket{1,1}$.\\
Now deuterium is build up of a proton and a neutron, thus it should have $\ket{\frac{1}{2},\frac{1}{2}}\otimes\ket{\frac{1}{2},-\frac{1}{2}} = \sqrt{\frac{1}{2}}\ket{1,0} + \sqrt{\frac{1}{2}}\ket{0,0}$
 Coupling $\pi^0$ and deuterium gives:
 \begin{equation}
 	\bra{1,0}\otimes\left(\sqrt{\frac{1}{2}}\ket{1,0} + \sqrt{\frac{1}{2}}\ket{0,0}\right) = \sqrt{\frac{1}{3}}\ket{2,0} + \sqrt{\frac{1}{2}}\ket{1,0} - \sqrt{\frac{1}{6}}\ket{0,0}
 \end{equation}
 And coupling $\pi^+$ and deuterium:
\begin{equation}
 \bra{1,1}\otimes\left(\sqrt{\frac{1}{2}}\ket{1,0} + \sqrt{\frac{1}{2}}\ket{0,0}\right) = \sqrt{\frac{1}{4}}\ket{2,1} + \sqrt{\frac{1}{4}}\ket{1,1} + \sqrt{\frac{1}{2}}\ket{1,1}
\end{equation}
So we have that:
\begin{equation}
	\frac{\sigma(\pi^0d \rightarrow pn)}{\sigma(\pi^+d \rightarrow pp)}\propto
	\frac{|(\sqrt{\frac{1}{3}}\bra{2,0} + \sqrt{\frac{1}{2}}\bra{1,0} - \sqrt{\frac{1}{6}}\bra{0,0})\cdot(\sqrt{\frac{1}{2}}\ket{1,0} + \sqrt{\frac{1}{2}}\ket{0,0})|^2}{|(\sqrt{\frac{1}{4}}\bra{2,1} + \sqrt{\frac{1}{4}}\bra{1,1} + \sqrt{\frac{1}{2}}\bra{1,1})\cdot\ket{1,1}|^2} 
	= \frac{\frac{1}{3} - \sqrt{\frac{1}{12}}}{\frac{3}{4} + \sqrt{\frac{1}{2}}}
\end{equation}
\subsection{$\Delta^+(1232)$-decays}
$\Delta^+(1232)$  has an isospin and $I_3$ described by $\bra{\frac{3}{2},\frac{1}{2}}$. Coupling n and $\pi^+$ gives: $\ket{\frac{1}{2},-\frac{1}{2}}\otimes\ket{1,1} = \sqrt{\frac{1}{3}}\ket{\frac{3}{2},\frac{1}{2}} + \sqrt{\frac{2}{3}}\ket{\frac{1}{2},\frac{1}{2}}$ and coupling p and $\pi^0$ gives: $\ket{\frac{1}{2},\frac{1}{2}}\otimes\ket{1,0} = \sqrt{\frac{2}{3}}\ket{\frac{3}{2},\frac{1}{2}} - \sqrt{\frac{1}{3}}\ket{\frac{1}{2},\frac{1}{2}}$. \\\\
We hereby see that there is a 33.33\% chance that $\Delta^+(1232)$ decays into n$\pi^+$:
\begin{equation*}
\left|\bra{\frac{3}{2},\frac{1}{2}}\left(\ket{\frac{1}{2},-\frac{1}{2}}\otimes\ket{1,1}\right)\right|^2 = \left|\bra{\frac{3}{2},\frac{1}{2}}\left(\sqrt{\frac{1}{3}}\ket{\frac{3}{2},\frac{1}{2}} + \sqrt{\frac{2}{3}}\ket{\frac{1}{2},\frac{1}{2}}\right)\right|^2 = \frac{1}{3}
\end{equation*}
and a 66.67\% chance that it decays into p$\pi^0$:
\begin{equation}
	\left|\bra{\frac{3}{2},\frac{1}{2}}\left(\ket{\frac{1}{2},\frac{1}{2}}\otimes\ket{1,0}\right)\right|^2 = \left|\bra{\frac{3}{2},\frac{1}{2}}\left(\sqrt{\frac{2}{3}}\ket{\frac{3}{2},\frac{1}{2}} - \sqrt{\frac{1}{3}}\ket{\frac{1}{2},\frac{1}{2}}\right)\right|^2 = \frac{2}{3}
\end{equation}
\subsection{propability proton spin parallel to $\Delta$ in $\Delta^+(1920) \rightarrow p\rho^0$}
When a $\Delta^+(1920)$ decays to a proton and $\rho_0$-meson it can do this in 2 ways: sending out the meson parallel to $\Delta$ and leaving the proton anti-parallel or sending the meson out anti-parallel to $\Delta$. As the spin of the $\rho_0$-meson is 1 a decay in which $\rho_0$ is parallel to the spin of the proton results in a proton with spin 1/2 anti-parallel to the original $\Delta$ while an anti-parallel decay results in a proton with spin 3/2 parallel to the $\Delta$ which is not possible, as such $\Delta^+(1920)$ only decays with the resulting proton being anti-parallel.
\subsection{Pion-nucleon scattering}
\begin{eqnarray*}
	\pi^+p &\rightarrow & \pi^+p\\
	\pi^-p &\rightarrow & \pi^-p\\
	\pi^-p &\rightarrow & \pi^0n
\end{eqnarray*}
If at a certain energy the reaction proceeds through an intermediate $I=3/2$ state i.e a $\Delta$ state then we have to select the coupling with $I=3/2$:
\begin{itemize}
	\item Coupling $\pi^+$ and p gives: $\ket{1,1}\otimes\ket{\frac{1}{2},\frac{1}{2}} = \ket{\frac{3}{2},\frac{3}{2}}$
	\item Coupling $\pi^-$ and p gives: $\ket{1,-1}\otimes\ket{\frac{1}{2},\frac{1}{2}} = \sqrt{\frac{1}{3}}\ket{\frac{3}{2},-\frac{1}{2}} - \sqrt{\frac{2}{3}}\ket{\frac{1}{2},-\frac{1}{2}}$
	\item Coupling $\pi^0$ and n gives: $\ket{1,0}\otimes\ket{\frac{1}{2},-\frac{1}{2}} = \sqrt{\frac{2}{3}}\ket{\frac{3}{2},-\frac{1}{2}} + \sqrt{\frac{1}{3}}\ket{\frac{1}{2},-\frac{1}{2}}$
\end{itemize}
Now we can see from the feynmann diagrams that:
\begin{eqnarray*}
	\pi^+p &\rightarrow & \Delta^{++} \rightarrow \pi^+p\\
	\pi^-p &\rightarrow & \Delta^0 \rightarrow \pi^-p\\
	\pi^-p &\rightarrow & \Delta^0 \rightarrow \pi^0n
\end{eqnarray*}
And thus:
\begin{itemize}
	\item $\frac{\sigma(\Delta^{++} \rightarrow \pi^+p)}{\sigma(\Delta^0 \rightarrow \pi^-p)}\propto
	\frac{|\bra{\frac{3}{2},\frac{3}{2}}\cdot(\ket{1,1}\otimes\ket{\frac{1}{2},\frac{1}{2}})|^2}{|\bra{\frac{3}{2},-\frac{1}{2}}\cdot(\ket{1,-1}\otimes\ket{\frac{1}{2},\frac{1}{2}})|^2}
	=\frac{|\bra{\frac{3}{2},\frac{3}{2}}\cdot(\ket{\frac{3}{2},\frac{3}{2}})|^2}{|\bra{\frac{3}{2},-\frac{1}{2}}\cdot(\sqrt{\frac{1}{3}}\ket{\frac{3}{2},-\frac{1}{2}} - \sqrt{\frac{2}{3}}\ket{\frac{1}{2},-\frac{1}{2}})|^2}
	=\frac{1}{1/3} = 3$
	\item Analogously $\frac{\sigma(\Delta^0 \rightarrow \pi^+p)}{\sigma(\Delta^0 \rightarrow \pi^0n)}=\frac{1}{2/3} = \frac{3}{2}$
	\item And $\frac{\sigma(\Delta^0 \rightarrow \pi^-p)}{\sigma(\Delta^0 \rightarrow \pi^0n)}=\frac{1/3}{2/3} = \frac{1}{2}$
\end{itemize}
Now for a nucleon state ($I=1/2$)  we can do the same procedure but project to $\bra{\frac{1}{2},\pm\frac{1}{2}}$, this gives us:
\begin{itemize}
	\item $\frac{\sigma(N \rightarrow \pi^+p)}{\sigma(N \rightarrow \pi^-p)} = 0$
	\item $\frac{\sigma(N \rightarrow \pi^+p)}{\sigma(N \rightarrow \pi^0n)} = 0$
	\item $\frac{\sigma(N \rightarrow \pi^-p)}{\sigma(N \rightarrow \pi^0n)} = \frac{\sigma(n \rightarrow \pi^-p)}{\sigma(n\rightarrow\pi^0n)}=
	\frac{|\bra{\frac{1}{2},-\frac{1}{2}}(\sqrt{\frac{1}{3}}\ket{\frac{3}{2},-\frac{1}{2}}-\sqrt{\frac{2}{3}}\ket{\frac{1}{2},-\frac{1}{2}})|^2}{|\bra{\frac{1}{2},-\frac{1}{2}}(\sqrt{\frac{2}{3}}\ket{\frac{3}{2},-\frac{1}{2}} + \sqrt{\frac{1}{3}}\ket{\frac{1}{2},-\frac{1}{2}})|^2} = 2$
\end{itemize}

\section{Baryon structure}
In the ground state band, the total angular momentum $J$ is entirely dependent on de spins of the quarks. If we name $\vec{S}_{12} = \vec{S}_1 + \vec{S}_2$ (analogously to $\vec{L}_{12}$), we have that:
\begin{equation*}
	\vec{S}_{12} = \vec{S}_1\otimes\vec{S}_2 = \ket{\frac{1}{2},m_1}\otimes\ket{\frac{1}{2},m_2}
\end{equation*}
Now we have that, if you couple two angular momenta $j_1$ and $j_2$ ($j=j_1+j_2$), j can take on the values\footnote{This is only true if pauli's exclusion principle is satisfied, which is the case here because of color charge}:
\begin{equation}
	j = j_1+j_2,j_1+j_2-1,...,|j_1-j_2|
\end{equation}
And the z-projection m can be:
\begin{equation}
	m = -j,-j+1,...,j-1,j
\end{equation}
Now applying this to $\vec{S}_1\otimes\vec{S}_2$ we get that $S_{12}$ can take on the values 1 and 0, and thus the states
\begin{equation*}
	\ket{0,0};\ket{1,-1};\ket{1,0}\ket{1,1}
\end{equation*}
Now coupling $\vec{S}_{12}$ and $\vec{S}_3$ to $\vec{S}$ we get that S can take the values $\frac{1}{2}$ and $\frac{3}{2}$, giving us the following possible states:
\begin{equation}
	\ket{\frac{1}{2},\pm\frac{1}{2}};\ket{\frac{3}{2},\pm\frac{1}{2}}\text{ and }\ket{\frac{3}{2},\pm\frac{3}{2}}
\end{equation}
As L=0 we have that the parity of these states is $(-1)^L = 1$, i.e a positive parity +. We thus have the states
\begin{equation}
	^2S_{\frac{1}{2}}^+ \text{ and } ^4S_{\frac{3}{2}}^+
\end{equation}
Written as $^{(2S+1)}\boldsymbol{L}_J^p$.\\\\
Now for L=1 (negative parity) we have to couple one more time resulting in $J = \frac{3}{2},\frac{1}{2}$ for $S=\frac{1}{2}$ and $J = \frac{5}{2},\frac{3}{2},\frac{1}{2}$ for $S=\frac{3}{2}$, or equivalently the states:
\begin{equation}
	^2P^-_{3/2}\text{ ; }^2P^-_{1/2}\text{ ; }^4P^-_{5/2}\text{ ; }^4P^-_{3/2}\text{ ; }^4P^-_{1/2}
\end{equation}
\bibliography{sources}
\bibliographystyle{plain}
\end{document}
